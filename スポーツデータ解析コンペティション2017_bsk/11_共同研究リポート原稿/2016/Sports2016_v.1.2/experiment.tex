\section{類似プレーの抽出実験}
\label{sec:exp}

\subsection{適用条件}
第\ref{sec:input}章で作成した全文書データに対して、LDAを適用し、サッカーの攻撃パターンの分類を行う。
LDAパラメータとして、トピック数$K=5$、ハイパーパラメター$\bf{\alpha},\bf{\beta}$の事前分布には無情報一様分布を設定した。
高頻度・低頻度単語の削除は行わず、生成したすべての単語を分析対象とした。
なお、パラメータ更新が収束した場合、もしくは50回繰り返し計算を実施した場合に計算終了とする。

\subsection{推定されたトピックの解釈}
以上の設定のもとで推定された5つのトピックにおいて特徴的な単語を示し、各トピックにおける攻撃パターンの意味合いを示す(表 \ref{tbl:topic})。
作成に当たっては以下の二つを参考にした。
一つ目はトピックごとの単語分布$p(w|z_i),i=\{1,2,3,4,5\}$と、デフォルト単語分布$p(w)$(すなわち、データ全体における正規化単語ヒストグラム:トピックが1つの場合の単語分布)との比を単語ごとに算出し、降順に並べたものである。
二つ目は、単語条件付きトピック分布$p(z_i|w),i=\{1,2,3,4,5\}$である。
%表は添付するパワポのp2、p3(p3はもとのp21と同一)

\begin{table*}[t]
  \centering
  \caption{推定されたトピックの解釈}
  \label{tbl:topic}
  \begin{tabular}{|c|l|l|l|}
  \hline
	Topic No. & トピックの意味合い & 特徴的な単語  \\ \hline \hline
  
	1 & 敵陣深くまで攻めきる & \begin{tabular}[c]{@{}l@{}} シュート / クロス / スルーパス / ドリブル / CK \\ ボール位置敵陣深く(hotzone1-12付近) \\ 相手選手多い(neighborplayerに2) / 守備ライン低い(50m付近)
\end{tabular} \\ \hline

	2 & \begin{tabular}[c]{@{}l@{}} その他の自陣での展開と\\ 短時間で終了した攻撃 \end{tabular} &  ボール位置自陣(hotzone31-42付近) / その他特徴的単語は少ない \\ \hline
  
	3 & \begin{tabular}[c]{@{}l@{}} アタッキングサード \\ 進入の攻防 \end{tabular} & \begin{tabular}[c]{@{}l@{}}オフサイド / ボール位置バイタル(hotzone7-24) \\ 守備ライン中盤(30m付近)
\end{tabular} \\ \hline 
    
	4 & \begin{tabular}[c]{@{}l@{}} 相手守備ブロック形成時の \\ ビルドアップ \end{tabular} & \begin{tabular}[c]{@{}l@{}} クリア / GK / フィード / 前方へのパス(失敗含む) \\ ボール自陣(hotzone31-42付近) / 脆弱度低い (2\% 以下)
\end{tabular}\\ \hline

	5 & \begin{tabular}[c]{@{}l@{}} ポジティブトランジションと \\ カウンター \end{tabular} & \begin{tabular}[c]{@{}l@{}} ブロック / タックル / インターセプト / シュート / スルーパス \\ ドリブル / オフサイド / フィード / 被ファウル \\ ボール自陣深く(hotzone43-54) / 相手選手多い(neighborplayerに2) \\ 守備ライン高い(0m付近) / 脆弱度高い(10\% 近く)
\end{tabular}\\ \hline

  \end{tabular}
\end{table*}



作成した単語を用いてサッカーの攻撃プレーを表現した結果、おおむねサッカーらしいトピック5つに自動分類できたと考えられる。
たとえば、トピック1はプレー位置が敵陣深くで、守備側のラインも深い。
さらに、ボール周囲に相手選手が多く、シュートやクロスも特徴的な単語となった。
また、トピック5ではボール位置が自陣深く、相手の守備ラインが高く脆弱度も高い一方で、スルーパスやシュート、被ファウルのような単語も特徴となった。
これらの結果は、攻撃をシュートで終えるためには、敵陣深くからのスルーパスやクロスまたはカウンターが有効であることを示唆している。
これは、一般に想定されるサッカーの知識と類似している。
また、サッカーの多くの時間はトピック4に表れる自陣でのビルドアップやその阻止、またトピック3に表れるアタッキングサード進入の攻防であることや、それ以外の時間が展開のない自陣でのボールポゼッション(トピック2)であることもうかがえる。



\subsection{実際の例}
%%%%%下に続くであろう類似プレー推薦に使う例を出したほうがいいと思うので、お任せします。
%イメージはもともとのパワポp22の下半分。
次に、トピックの分類結果を実際のプレー内容と照らし合わせた上で検証する。
文書条件付きトピック分布$p(z_i|d),i=\{1,2,3,4,5\}$を参照しながら、ほとんど一つのトピックのみで構成されている文書と、複数のトピックで構成されている文書の2パターンに関して、実際のプレー内容とトピック内容を比較する。

例えば、トピック1が90\%程度の割合で占めている例では、浦和がボールを前進させシュートまで繋げている様子が確認できた。
また、トピック5が90\%程度の割合で占めている例では、鹿島がカウンターで40m進み被ファールを受けた様子が確認できた。
さらに、構成トピックが混合しているものの例として(トピック1:3:5 = 3:4:2)、川崎の一連の攻撃を挙げられる。
この例では、川崎が自陣中ほどで相手のパスをカットし、比較的短時間のうちにバイタルエリア付近、さらには敵陣深くまでパスを繋いでいる様子が伺えた。

これらの例から、単語条件付きトピック分布$p(z_i|w),i=\{1,2,3,4,5\}$から推定したトピックごとの攻撃パターンは、文書条件付きトピック分布$p(z_i|d),(i=\{1,2,3,4,5\})$をもとに抽出した実際の攻撃内容から推察される攻撃パターンとある程度一致していることが分かった。
特に、複数のトピックの混合で文書条件付きトピック分布が構成されている場合においても、構成されているトピック(攻撃パターン)のそれぞれの特徴がプレー内容に反映されていることが伺えた。


\subsection{類似プレーの抽出結果}
最後に、文書条件付きトピック分布$p(z_i|d)$の類似度に基づく類似プレーの抽出結果を示す。
ただし、全プレーに対して類似プレーの自動抽出を行い、その結果が「似ている」か「似ていない」の判断を下すのは不可能であるため、ここではいくつかの例で検証する。

具体的には、前節で紹介した表\ref{tbl:res1}におけるNo.7~9の攻撃について類似した攻撃の検索を試みる。
検索対象の攻撃$d_{req}$との文書条件付きトピック分布$\bvec{\theta}_{d_{req}}$と、他のすべての攻撃$d_{db}$の文書条件付きトピック分布$\bvec{\theta}_{d_{db}}$のコサイン類似度$\cos(\bvec{\theta}_{d_{req}},\bvec{\theta}_{d_{db}})$を計算し、高いスコアを算出した攻撃を抽出する。
ただし、一連の攻撃が少数のアクションのみによって構成されている場合、効果的な攻撃でなく検索結果として好ましくない可能性や、攻撃内容が複雑ではないため、構成されるアクション(CK、シュート)だけで検索できてしまうと思われる。
そこで、検索条件として、「15回以上のアクションによって構成されている攻撃」と設定することとした。

類似プレーの抽出結果を表\ref{tbl:res2}に示す。
例えば攻撃番号No.9の検索結果に対してみてみると、フィードやスローインによってバイタルエリア付近までボールを運び、その後パスを繋いでいるという攻撃の様子に共通点が見られた。
ただし、検索結果の中には似ているか似ていないかの判断が難しかったり、そもそも攻撃パターンの解釈自体は難しかったりするものも多く存在した。
抽出された攻撃パターンがユーザにとって好ましいものであったのかなど、別途議論すべきであると考えられる。


\begin{table*}[t]
  \centering
  \caption{文書条件付きトピック分布と実際のプレーとの関係}
  \label{tbl:res1}
  \begin{tabular}{|c|l|l|l|}
  \hline
	No. & \begin{tabular}[c]{@{}l@{}}文書条件つきトピック分布\\ (各文書におけるトピック構成比)\end{tabular} & 実際のプレー内容の概要  \\ \hline \hline
  
	1 & \textbf{0.9} : 0.1 : 0.0 : 0.0 : 0.0 & 浦和がボールを前進させシュートまで繋げる \\ \hline

	2 &\textbf{1.0} : 0.0 : 0.0 : 0.0 : 0.0 & \begin{tabular}[c]{@{}l@{}}神戸が自陣深くでクリアしたボールをパスで前線まで運び、\\サイドをドリブルで切り込む\end{tabular} \\ \hline
  
	3 & 0.0 : \textbf{1.0} : 0.0 : 0.0 : 0.0 & \begin{tabular}[c]{@{}l@{}}川崎がハーフライン奥でボールを取得\\MFを中心にペナルティーエリア手前でパスを回す展開\end{tabular} \\ \hline 
    
	4 & 0.0 : 0.0 : \textbf{1.0} : 0.0 : 0.0 & \begin{tabular}[c]{@{}l@{}}鳥栖が敵陣中ほどでボールを取得し、\\ MFを中心にペナルティーエリア手前でサイドチェンジを展開\end{tabular}\\ \hline

	5 & 0.0 : 0.0 : 0.0 : \textbf{1.0} : 0.0 & \begin{tabular}[c]{@{}l@{}}湘南が自陣の浅い位置でスローインを取得し、\\DFまでパスを下てからパスを繋いでいき、\\敵陣へと進入したところで被ファウル\end{tabular}\\ \hline

	6 & 0.1 : 0.0 : 0.0 : 0.0 : \textbf{0.9} & 鹿島がカウンターで40m進み、ファールを受ける\\ \hline

	7 & \textbf{0.4} : 0.0 : \textbf{0.6} : 0.0 : 0.0 & \begin{tabular}[c]{@{}l@{}}広島がキーパーからのフィードをパスを繋ぎ、\\敵陣深くでスルーパスで終える\end{tabular}\\ \hline

	8 & \textbf{0.6} : 0.0 : 0.0 : 0.0 : \textbf{0.4} & \begin{tabular}[c]{@{}l@{}}川崎が自陣深くで相手のパスをブロック、ポストプレイ\\キーパーを経由して短い手数で前線に運び、最後はクロスで終わる\end{tabular}\\ \hline

    9 & \textbf{0.3} : 0.0 : \textbf{0.4} : 0.0 : \bf{0.2} & \begin{tabular}[c]{@{}l@{}}川崎が自陣中ほどで相手のパスをカットし、\\ 比較的短時間のうちにバイタルエリア付近、\\ さらには敵陣深くまでパスを繋ぐ\end{tabular}\\ \hline
    
  \end{tabular}
\end{table*}


\begin{table*}[t]
  \centering
  \caption{文書条件付きトピック分布と実際のプレーとの関係}
  \label{tbl:res2}
  \begin{tabular}{|c|l|l|l|}
  \hline
	No. & コサイン類似度 & 実際のプレー内容の概要  \\ \hline \hline
  
	\multicolumn{3}{|l|}{検索対象:No.7 in Table\ref{tbl:res1}} \\ \hline
  
	7-1 & 0.99 & 川崎がキーパーからのフィードをパスを繋ぎ、敵陣深くでスルーパス \\ \hline
    7-2 & 0.99 & 川崎がキーパー発のFKからパスを繋ぎ、敵陣深くでスルーパス \\ \hline \hline
    
    \multicolumn{3}{|l|}{検索対象:No.8 in Table\ref{tbl:res1}} \\ \hline
          
    8-1 & 0.99 & \begin{tabular}[c]{@{}l@{}}湘南が自陣の浅い位置でインターセプトし、敵陣深くまで攻める\\守備側のクリアなどのアクションあり\end{tabular} \\ \hline \hline
    
	\multicolumn{3}{|l|}{検索対象:No.9 in Table\ref{tbl:res1}} \\ \hline

	9-1 & 0.99 & 名古屋がキーパーからのフィードを起点にボールを前線へと繋げる \\ \hline
    
    9-2 & 0.95 & 鹿島がピッチ中ほどからのスローインを起点にパスを繋ぎ、最後にはクロスを上げる \\ \hline

    
  \end{tabular}
\end{table*}






