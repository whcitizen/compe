%%%%%%%%%%%%%%%%%%%
%% プリアンサンブル
\documentclass[10pt,a4j]{jarticle}
\usepackage{amsmath}
\usepackage[dvipdfmx]{graphicx}
\usepackage{subfigure}
\usepackage{here}
\usepackage[small, bf]{caption}

%maketitleの再設定
\usepackage[top=15truemm,bottom=15truemm,left=15truemm,right=15truemm]{geometry}
\makeatletter
 \renewcommand{\@maketitle}{\newpage
  %\null%%
  %\vskip 5.0em
 \begin{center}
   {\vspace*{0.1mm} \LARGE \@title \par} \vskip 1.5em 
   %{\normalsize \lineskip .5em \begin{tabular}[t]{l}\@author \end{tabular}\par}
   \vskip 1em {\large \@date} 
   \end{center}%%%
 \begin{flushright}%%%
  {\normalsize \lineskip .5em \begin{tabular}[t]{l}\@author \end{tabular}\par}
  \end{flushright}
 %\end{center}%%%\end{center}
 \par
 \vskip 1.5em}
 \makeatother

%thebibliographyの再設定
\makeatletter
  \renewenvironment{thebibliography}[1]{%
    \global\let\presectionname\relax
    \global\let\postsectionname\relax
    \subsection*{\refname}\@mkboth{\refname}{\refname}%%%%<-section
    \list{\@biblabel{\@arabic\c@enumiv}}%
          {\settowidth\labelwidth{\@biblabel{#1}}%
          \leftmargin\labelwidth
          \advance\leftmargin\labelsep
          \setlength\baselineskip{10pt}% 文字サイズ
          \setlength\itemsep{0zh}% アイテム間のマージン
          \@openbib@code
          \usecounter{enumiv}%
          \let\p@enumiv\@empty
          \renewcommand\theenumiv{\@arabic\c@enumiv}}%
    \sloppy
    \clubpenalty4000
    \@clubpenalty\clubpenalty
    \widowpenalty4000%
    \sfcode`\.\@m}
    {\def\@noitemerr
      {\@latex@warning{Empty `thebibliography' environment}}%
    \endlist}
\makeatother

%二段組の設定
\usepackage[]{multicol}
\setlength{\columnsep}{10truemm}

%章番号の終わりをドットで繋げる
\usepackage{secdot}
\sectiondot{subsection}
%bold vectorの定義
\newcommand{\bvec}[1]{\mbox{\boldmath $#1$}}
%ページ番号の消去
\pagestyle{empty} % すべてのページ番号を消去


%%%%%%%%%%%%%%%%%%
%% 文書設定 
\title{トピックモデルを用いたサッカーの攻撃パターンの分類による\\類似プレーの抽出}
\author{神谷 啓太(東京大学大学院 工学系研究科 社会基盤学専攻)\\
中西 航(東京工業大学 環境・社会理工学院 土木・環境工学系)\\
泉 裕一朗(東京大学大学院 工学系研究科 社会基盤学専攻)\\
〒113-8656 東京都文京区本郷7-3-1 TEL:03-5841-6118\\
E-mail:kamiya@trip.t.u-tokyo.ac.jp\\}
\date{\empty}


%%%%%%%%%%%%%%%%%%%
%% 原稿開始 
\begin{document}
%タイトル表示
\maketitle
\thispagestyle{empty} % 1ページ目を表示させない
%2段組本文の開始
\begin{multicols}{2}
%はじめに
\section{はじめに}
\label{sec:hajimeni}

%%% 以下、まだ適当に。去年のレポートやパワポのコピペ。
数多くのスポーツで、センサやデータ計測員を駆使して多量のデータを取得し、統計的手法によってそれらを分析し、得られた示唆や知見をチーム戦略や選手評価に活用することが一般的となっている。
サッカーにおいても、ボール支配率やパス・シュート本数などの従来のスタッツを始め、ここ数年では選手の走った経路を逐次記録したトラッキングデータなどが計測されている\cite{foot}。


豊富なデータの中から、似たような「一連の攻撃」を探したいというタスクは、スタッフ、選手、観客などがプレーを振り返る際に重要である。
現在はデータ計測スタッフやマネージャー部員が逐一、「シュート」や「FK」というアクションタグや、「カウンター」や「クロス」などのプレー内容のフラグを付加している。
類似プレーを検索する際には、これらのタグ情報をもとに目的のプレーを抽出できることもあるが、それだけでは限界がある。
アクション以外にも「ピッチのどこで」「どんな選手配置の中」など一連の攻撃を説明するような検索条件はたくさん考えられるからである。
ただしそれらをひとつひとつ条件立てて検索するのは困難であり、類似プレー抽出の自動化が望まれる。


一連の攻撃を説明するような変数として、「起こったアクション」 「ボール位置」、「選手配置」など多様な変数が考えられる。
これらを変数としたクラスタリングを実施すれば類似プレーの抽出は可能であるが、同じ位置で、同じ順番に、同じ所要時間で展開するプレーはほぼ一つだけであり、複雑な時系列データすべてを見ても類似度は定義しにくい。
すなわち、スパース性の問題が生じる可能性がある。
また、時空間をある程度離散化してスパース性を解消しても、ひとつの攻撃の長さがそれぞれ異なるという問題が残り、異なる長さのベクトル間の類似度を測定することは難しい。

本研究では、トピックモデルを使用することで、長さの異なる文書でも、一つの共通したベクトルであるトピック分布として再表現できることに着目する。
すなわち、そのトピック分布を類似度に使うことで上記の問題を解決することができると考えられる。
本研究の目的は、トピックモデルを用いてサッカーの攻撃パターンの分類を行い、類似プレーの自動抽出を行うことである。
第\ref{sec:lda}章でトピックモデルの中でもLDA(Latent Dirichlet Allocation)について概説した後、第\ref{sec:input}章で分析に用いる変数を選定する。
第\ref{sec:exp}章で実際のプレーデータに対してLDAを適用し、推定されたトピックに関する解釈を行う。
最後に、本研究のまとめと今後の課題を第\ref{sec:owarini}章でまとめる。


%Latent Dirichlet Allocation
\section{トピックモデルを用いた類似プレーの抽出手法}
\label{sec:lda}
本章ではトピックモデルを用いた類似プレーの抽出手法について説明する。
まず、提案手法を構築するにあたり必要となるトピックモデル、LDA(Latent Dirichlet Allocation)について概説した後、サッカーにおける一連の攻撃をいかにトピックモデル的な観点から解釈可能かを議論する。
その後、トピックモデルを用いたサッカーのプレー内容の分類およびそれに基づく類似プレーの抽出手法について説明する。
なお、LDAについての解説は参考文献\cite{lda}\cite{topic_blue}\cite{topic_white}をもとに記す。

\subsection{Latent Dirichlet Allocation}
LDAは、もともと文書の確率的生成モデルとして提案されたモデルである。
ただし、文章の順序は無視し、Bag of Words(BoW)表現と呼ばれる単語と出現頻度のペアの集合をモデル化する。
ここで、BoW表現は単語が共起している現象を表しており、文書集合として考えると統計的に共起しやすい単語の集合がいくつか存在するはずである。
LDAは、この単語の共起性を統計モデルとして数理的に扱うために提案された。


\subsubsection{LDAの生成過程}
LDAでは、文書中の単語がどの潜在トピックによって生成されたかを示す潜在変数を導入する。
具体的には、文書$d$の$i$番目の単語を$w_{d,i}$として、対応する潜在変数を$z_{d,i}$と定義する。
潜在トピックの添字集合を$\{1,2,...,K\}$とする($z_{d,i}\in\{1,2,...,K\}$)。
各潜在変数の値は、それぞれ単語の出現分布$\phi_k(k=1,2,...,K)$に対応する。
つまり、文書中の各単語は離散値をとる潜在変数を背後に保持しており、その潜在変数の値が同じ単語はトピックに属する(同じ単語の出現分布に従う)というモデル化を行う。

文書数を$M$、文書$d$の文章長(総単語数)を$n_d$とする。
LDAでは、文章は複数のトピックから構成され、その構成比を離散分布としてもつ。
$\theta_{d,k}$を、文章$d$でトピック$k$が出現する確率(文章$d$でのトピック$k$の構成比率)とし、トピック分布を$\bvec{\theta}_d=(\theta_{d,1},\ldots,\theta_{d,K})$とする。
$\phi_{d,v}$をトピック$k$における単語$v$の出現確率とし、単語の出現分布を$\bvec{\phi}_k=(\phi_{k,1},\ldots,\phi_{k,V})$とする。
$\bvec{\theta}_d$や$\\bvec{phi}_k$は確率ベクトルであるので、確率ベクトル上の確率分布であるDirichlet分布(Dirと表記)による生成を仮定する。
すなわち、
\begin{equation}
	\bvec{\theta}_d \sim Dir(\bvec{\alpha}), d=1,\ldots,M
\end{equation}
\begin{equation}
	\bvec{\phi}_k \sim Dir(\bvec{\beta}), k=1,\ldots,K
\end{equation}

ここで、$\bvec{\alpha}=(\alpha_1,...,\alpha_K)$は$K$次元ベクトル、$\bvec{\beta}=(\beta_1,...,\beta_V)$は$V$次元ベクトルで、いずれもDirichlet分布のパラメータである。

単語$w_{d,i}$や潜在トピック$z_{d,i}$は離散値なので、多項分布(Multiと表記)を生成分布として仮定する。
すなわち、各文書$d(=1,...,M)$において、各単語は以下の生成過程を仮定する。
\begin{equation}
	z_{d,i} \sim Multi(\bvec{\theta}_d)
\end{equation}
\begin{equation}
	w_{d,i} \sim Multi(\phi_{z_{d,i}}) ,i=1,...,n_d
\end{equation}

%この生成過程のグラフィカルモデルを図\ref{fig:lda}に示す。



\subsection{トピックモデルとしてみるサッカーのプレー}
自然言語処理では、以下の仮定に基づいてトピックモデルを適用する。
\begin{itemize}
	\setlength{\itemsep}{0cm} % 項目間
	\item 文はいくつかのトピック(話題)の重み付き和で構成される
	\item トピックごとに使われやすい単語の分布がある
	\item 文は、トピック→単語の順で構成する単語数をサンプリングしたものである
	\item 文の中での単語の順番は不問とする、すなわち、順不同の単語から文の意味が一意に定まるとする
\end{itemize}

一方、本研究ではサッカーのプレーに対してトピックモデルを適用するにあたり、上記に対応するような以下の仮定を置く。
\begin{itemize}
	\setlength{\itemsep}{0cm} % 項目間
	\item 一連の攻撃は、ボール奪取時の状況やチームの特性に応じたトピック(攻撃パターン)の重み付き和で表現される
	\item 攻撃パターンごとに選択されやすいアクションや起こりやすい選手配置がある
	\item 一連の攻撃は、攻撃パターン→アクション/配置の順番に攻撃を構成する要素数をサンプリングしたものである
	\item 一連の攻撃内でのアクション/配置の順番は不問とする、すなわち、順不同のアクション/配置からプレー状況が定まるとする
\end{itemize}

最後の仮定について補足する。
直感的には、サッカーの一連のプレーにおいてプレーが生起する順番は意味を持ちそうである。
しかしながら、本研究では、仮に単語を適切に設定できれば、ある単語の集合を考えたときに起こりうるプレーの順番は実質的に定まると想定する。
このことにより、時系列データが格段に扱いやすくなるという大きなメリットがあるためである。
ただし、データから分析に有用な単語を生成する方法自体の検討が必要となるため、これを次章で説明する。
なお、この仮定には議論の余地が残るが、後述の適用において一定程度の妥当性は示されたと考えている。

\subsection{類似手法の抽出手法}
トピックモデルを用いて攻撃パターンを分類し、その分類結果に基づく類似プレーの抽出手法を以下の通り提案する。
\begin{enumerate}
	\setlength{\itemsep}{0cm} % 項目間
	\item ボールタッチ/トラッキングデータより文書を作成する
	\item 検索対象の文書$d_{req}$のトピック分布$\bvec{\theta}_{d_{req}}$を推定する
	\item データベース中の攻撃データ$d_{db}$とコサイン類似度$cos⁡(\bvec{\theta}_{d_{req}},\bvec{\theta}_{d_{db}})$ を計算
	\item その他検索条件(攻撃長、特定プレーの存在など)に基づき、類似度の高いものから抽出
\end{enumerate}





%入力変数の検討
\section{入力変数の検討}
\label{sec:input}

\subsection{使用したデータの説明}
本研究で使用したデータは、2015明治安田生命J1リーグ1stステージ第2節鹿島アントラーズ対湘南ベルマーレ戦および第17節松本山雅FC対湘南ベルマーレ戦の計2試合に関して、1/30秒毎にパスやタックルなどボール周辺で発生したイベントおよびその発生時刻と位置を取得したボールタッチデータと、1/25秒毎に選手及び審判のピッチ上での位置を取得したトラッキングデータの2種類である。
なお、これらのデータはデータスタジアム株式会社から提供を受けたものである。


\begin{figure*}[htbp]
  \begin{center} %センタリングする
    \includegraphics[width=18cm]{img/variable.eps}
    \renewcommand{\baselinestretch}{1}%
    \caption{入力変数の概念図. (a)前線位置。両チームの選手がなす支配領域が均衡する前線の$X$座標に関する平均値。図中の白色破線に対応。(b)コンパクトネス。一番前方の選手と後方2番目の選手の$X$座標上の距離。図中の灰色と白色の四角形の幅にそれぞれ対応。(c)守備脆弱度。自軍のオフサイドラインより前方10m、後方5mの長方形のうち、最寄りの味方選手から5m以上離れており、最近傍選手が相手選手であるような地点の合計面積の割合。図中の灰色で囲まれている図形の合計面積が占める割合に相当。}
    \label{fig:var} %ラベルをつけ図の参照を可能にする
  \end{center}
\end{figure*}


\subsection{入力変数の検討}
VARモデルの入力変数として用いる指標について検討を行う。
まず、戦況を表すのに十分な入力変数を用意できるよう、試合全体の流れを表現すると思われる指標や、各チームの攻勢・守勢を表す指標を複数作成した。
その後、変数間の相関分析を行い、共線性のない必要最低限な変数組が選定できるように配慮した。
その結果、ボール位置、前線位置、コンパクトネス、守備脆弱度、攻撃率の5種類の指標を分析に用いる変数として選定した。

以下、それらの指標について詳細な説明を行う。特に、前線位置、コンパクトネス、守備脆弱度については図\ref{fig:var}に概念図を示している。
なお、提供を受けたデータは1/25秒や1/30秒間隔であったが、VARモデルへの適用を踏まえ、分析に使うデータはすべて1秒間隔となるよう加工した。
また、ピッチ中央を原点とし、コートの長辺方向を$X$方向、短辺方向を$Y$方向と定義する。




\subsubsection{ボール位置}
試合全体の流れを表現する指標としてボール位置を選定した。攻勢・守勢の切り替わりによるボールの上下動のみを考慮するとし、$X$座標のみを採用した。

\subsubsection{前線位置}
両チームの選手がなす支配領域が均衡する$X$座標を前線位置と定義した。
この変数は全選手の動きを代表するため,各チームの攻勢と守勢に合わせ値が増減すると予測される。
既往研究\cite{kij}に従い、以下のとおり作成した。
\begin{enumerate}
\renewcommand{\labelenumi}{\roman{enumi})}
\setlength{\itemsep}{-5pt}
 \item チームで正負の異なるガウスカーネル(標準偏差3m)を各選手が時刻$t$に位置する座標に設定する。
 \item 全てのカーネルを足し合わせ,ピッチ上で値が0となる線分を前線とする。
 \item 前線の$X$座標に対する平均値を算出し,これを時刻$t$における前線位置とする。
 \item i)~iii)を全時刻について算出する。
\end{enumerate}

\subsubsection{コンパクトネス}
ホーム・アウェイチーム毎に、「一番前方に存在する選手の$X$座標」と「後方より2番目に存在する選手の$X$座標」の距離をコンパクトネスと定義する。
各チームの選手がその時刻において、どれだけピッチ上で展開できているかを示す指標であり、攻撃を展開するとコンパクトネスは大きくなり、守備に回るとコンパクトネスは小さくなると予想される。

\subsubsection{守備脆弱度}
チームの守備力の脆弱性を表す指標として、自陣の最終ライン付近において相手選手が侵入している程度を守備脆弱度と定義する。
具体的には図\ref{fig:var}(c)に示すように、「自軍のオフサイドラインより前方10m、後方5mの長方形のうち、最寄りの味方選手から5m以上離れており、最近傍選手が相手選手であるような地点の合計面積の割合」である。
守勢への転換直後には守備脆弱度は上昇し、守備陣営が整うと守備脆弱度は減少すると予想される。
一方、カウンターや人数をかけた攻撃時には守備脆弱度は上昇することが予想される。

\subsubsection{攻撃率}
ホームチームの直近10分間における攻撃頻度を表す指標として攻撃率を定義する。
ボールタッチデータ中に離散的に取得されている各チームの攻撃アクションを内挿し、1秒毎の連続的な値として算出した。







%適用結果
\section{戦況変化の抽出実験}
\label{sec:exp}

\subsection{適用条件}
ChangeFinderの入力変数として、ボール位置、前線位置、コンパクトネス(ホーム・アウェイチームごと)、守備脆弱度(ホーム・アウェイチームごと)、攻撃率の5種類・計7つのデータを7次元変数として設定した。
また、VARモデルの次数$K$は、オフラインでのVAR推定を行った場合の最小AIC値を参考に、$K=5$と決定した。
忘却率$r$や平滑化パラメータ$T,T′$についてはそれぞれ$r=0.01$、$T=50$、$T'=5$と設定した。
鹿島・湘南戦および松本・湘南戦の前後半のハーフ毎にChangeFinderを適用し、戦況変化の抽出を試みる。

なお、SDARアルゴリズムで用いるVARモデルの各種パラメータおよび統計量の初期値$\hat{\mu},\hat{\Sigma},C_i(i=1,...,K)$は、ハーフ開始60秒間で得られたデータから計算することとし、この間の変化点スコアは算出しない。

\subsection{適用結果および結果の考察}
松本・湘南戦後半を例として、図\ref{fig:res}の上段に入力時系列データ$\bvec{x}_t$を、下段にChangeFinderによる変化点スコアの出力値$Score(t)$を示す。
図\ref{fig:res}の上段において、1行目にボール位置と前線位置、2行目に湘南ベルマーレと松本山雅FCのコンパクトネス、3行目に湘南ベルマーレと松本山雅FCの守備脆弱度がそれぞれ灰色および黒色の実線で、そして4行目に松本山雅FCの攻撃率が示されている。
ボール位置と前線位置に関しては、正方向に値が上昇するにつれ、湘南ベルマーレの攻撃方向に進出していることを意味する。
下段は各時刻について算出された変化点スコア$Score(t)$であり、値が大きいほど戦況変化の度合いが大きいことを表している。

出力された変化点スコアを見ると、図中に(i)~(x)で示す合計10箇所で比較的大きな戦況変化が発生したと推定されたことが分かる。
紙面には掲載していないが鹿島・湘南戦前後半および松本・湘南戦前半の結果と比較すると、松本・湘南戦後半ではより高頻度で変化点が検出されていた。
松本・湘南戦後半では両チーム合わせ5つのゴールが生まれており、戦況の移り変わりが比較的大きかった試合であったことが、本結果において比較的高頻度に変化点が検出された理由として考えられる。

次に、検出された変化点が具体的にどういった戦況の変化を表しているのか、入力データの振る舞いと実際のプレーを比較することで結果の解釈および検証を行う。
ただし、図\ref{fig:res}からも分かるように入力データは複雑な動きをしているため、入力データから変化点検出の原因を直接分析するのは困難である。
そこで、パラメータ$\mu$の各時刻における推定値$\hat{\mu}$に着目する。
このパラメータはVARモデルの式(\ref{eq:var})より、自己回帰分を除いた平均的な値と解釈できる。
この推定値$\hat{\mu}$の振る舞いに着目し、検出された変化点との関係を分析することによって、どういった戦況の変化を表しているのか解釈を進めることが可能になると考える。

そこで、図\ref{fig:mu}に松本・湘南戦後半における推定値$\hat{\mu}$の推移と変化点の検出結果を示す。図中のグラフと変数の対応は図\ref{fig:res}と同様である。
図\ref{fig:res}で複雑な振る舞いをしていた入力値が、図\ref{fig:mu}の推定値$\hat{\mu}$では比較的滑らかな振る舞いをしていることが分かる。
また、推定値の時系列的振る舞いに変動があると、それ対応するように直後1分後あたりで変化点スコアが上昇していることが分かる。
なお、ChangeFinderでは第1段階および第2段階で移動平均処理を行っているため、観測データの入力から変化点の検出までに遅延が生じることに注意が必要である。

% 入力データと変化点スコアの出力
\begin{figure*}[htbp]
  \begin{center}
    \includegraphics[width=18cm]{img/res.eps}
    \renewcommand{\baselinestretch}{1}%
    \caption{松本・湘南戦後半における入力データおよび変化点スコア出力結果. 上段1行目にボール位置(ball)と前線位置(frontLine)、2行目に湘南ベルマーレと松本山雅FCのコンパクトネス(compact)、3行目に湘南ベルマーレと松本山雅FCの守備脆弱度(defence)をそれぞれ灰色および黒色で、4行目に松本山雅FCの攻撃率(attack)を示す。AWAYおよびHOMEはそれぞれ湘南ベルマーレと松本山雅FCに対応する。下段はChangeFinderによる変化点スコアの出力値$Score(t)$である。}
    \label{fig:res}
  \end{center}
\end{figure*}

%平均μと変化点スコアの出力および因果関係の考察
\begin{figure*}[htbp]
  \begin{center}
    \includegraphics[width=18cm]{img/mu.eps}
    \renewcommand{\baselinestretch}{1}%
    \caption{松本・湘南戦後半におけるパラメータ$\mu$の推定値$\hat{\mu}$の推移および変化点スコア出力結果. 図中のグラフと変数の対応は図\ref{fig:res}に同じ。なお、試合開始60秒間のパラメータは推定されていない。}
  \label{fig:mu}
  \end{center}
\end{figure*}

表\ref{tbl}に、各戦況変化点(i)~(x)について代表的な$\hat{\mu}$の変動の様子およびそれから想定される戦況変化の内容、ならびに実際にボールタッチデータから確認されたプレーの内容をまとめた。
例えば戦況変化点(i)の直前では、ボールが湘南攻撃方向に移動し、松本の守備脆弱度が上昇するという変動がパラメータ推定値$\hat{\mu}$の中で確認された。
この結果より、湘南が攻勢へと転換したことが戦況変化の内容として想定される。
実際にボールタッチデータより、湘南が複数のパスを繋ぎながら攻撃を展開している様子が確認できた。
他の変化点についても、表中にあるように、変化点スコア上昇の原因となるようなパラメータ推定値$\hat{\mu}$の変動から想定される戦況変化の内容が、実際のプレー内容と概ね合致していることが確認できた。

\begin{table*}[t]
  \centering
  \caption{検出された変化点の解釈および実際のプレーとの関係}
  \label{tbl}
  \begin{tabular}{|c|l|l|l|}
  \hline
  \# & 代表的な$\hat{\mu}$の変動 & 想定される戦況変化 & 実際に確認されたプレー  \\ \hline \hline
  i & \begin{tabular}[c]{@{}l@{}}ボールが湘南攻撃方向に移動\\ 松本の守備脆弱度が上昇\end{tabular} & 湘南の攻勢への転換 & 湘南がパスを繋ぎ攻撃を展開 \\ \hline
  i\hspace{-.1em}i & \begin{tabular}[c]{@{}l@{}}ボールが松本攻撃方向に移動\\ 湘南のコンパクトネスが低下\\ 松本のコンパクトネスが上昇\\ 湘南の守備脆弱度が上昇\\ 松本の守備脆弱度が低下\end{tabular} & 松本の攻勢への転換 & 松本がPKを獲得 \\ \hline
  i\hspace{-.1em}i\hspace{-.1em}i & \begin{tabular}[c]{@{}l@{}}湘南のコンパクトネスが上げ止まり\\ 松本のコンパクトネスが下げ止まり\\ 松本の守備脆弱度が回復\end{tabular} & 松本の守勢の実現 & 湘南がFKを獲得し松本が守備体勢に \\ \hline
  i\hspace{-.1em}v & \begin{tabular}[c]{@{}l@{}}ボールが松本攻撃方向に回復\\ 松本のコンパクトネスが上昇\end{tabular} & 松本の攻勢への転換 & 松本が自陣でFKを獲得しピンチを脱する \\ \hline
  v & \begin{tabular}[c]{@{}l@{}}松本のコンパクトネスが低下\\ 松本の守備脆弱度が上昇\\ 湘南の守備脆弱度が低下\end{tabular} & \begin{tabular}[c]{@{}l@{}}湘南の攻勢への転換\\ 松本の守勢の実現\end{tabular} & 湘南のCK/FKによる連続的な攻撃 \\ \hline
  v\hspace{-.1em}i & 湘南の守備脆弱度が回復 & 湘南の守備の実現 & 湘南得点後の松本キックオフ \\ \hline
  v\hspace{-.1em}i\hspace{-.1em}i  & 松本の守備脆弱度が低下 & 松本の守勢の実現 & 複数選手の交代により松本守勢の立て直しか \\ \hline
  v\hspace{-.1em}i\hspace{-.1em}i\hspace{-.1em}i & ボール位置が短時間に前後 & 試合の活性化 & 松本得点直後に湘南の得点 \\ \hline
  i\hspace{-.1em}x & \begin{tabular}[c]{@{}l@{}}ボールが松本攻撃方向に移動\\ 湘南の守備脆弱度が上昇\end{tabular} & 松本の攻勢への転換 & 松本のシュートを含む試合展開 \\ \hline
  x & \begin{tabular}[c]{@{}l@{}}ボールが松本攻撃方向に移動\\ 湘南の守備脆弱度が上昇\end{tabular} & 松本の攻勢への転換 & 同点弾を狙った松本の攻撃 \\ \hline
  \end{tabular}
\end{table*}

以上の結果を踏まえると、本分析方法により、攻勢・守勢への転換やセットプレーによる連続攻撃、連続得点による試合の活性化など、想定される戦況変化は概ね検出できていると考えられる。
ただし、実際の試合映像などを確認し、検出された戦況変化が妥当なものであるのか、また、未検出となった戦況変化等が存在しないかを専門的知見と共に検討を重ねたい。
また、現在は検出された変化点との関係をパラメータ$\mu$のみで見ているが、戦況のより高度な解釈に向け、$\mu$以外のパラメータについても精査を行う必要がある。
さらに、検出された戦況変化に対応するプレーにはセットプレーや選手交代などアウトオブプレーが多く含まれている。
実際、セットプレーを機に戦況が大きく変わることは多々あるが、変化点(i)や(x)などインプレー中の戦況変化の抽出も重要であろう。
インプレー中の戦況変化の抽出のためには、ChangeFinderに適用する時系列データからアウトオブプレー中のデータを排除する、一つのインプレーデータ毎にChangeFinderを適用するなど、分析方法を修正することが解決策として考えられる。




%おわりに
\section{おわりに}
\label{sec:owarini}

本研究では、トピックモデルを用いてサッカーの攻撃パターンを分類し、類似した一連の攻撃を自動で抽出する方法の提案を行った。
トピックモデルであるLDAを適用するにあたり、サッカーのトラッキングデータおよびボールタッチデータから一連の攻撃を説明するような単語の検討を行い、アクション名、ボール位置、選手配置(コンパクトネス × オフサイドライン)、守備脆弱度、ボール周辺選手配置の5種類の単語を作成し、攻撃内容を離散的な文書として表現した。
作成した文書に対してLDAを適用し、トピックごとの単語分布とデフォルト単語分布の比較、および単語条件付きトピック分布の参照を通じて各トピックの意味合いを解釈した結果、おおむねサッカーらしいトピック5つに自動分類できたと考えられる。
また、実際のプレー内容とトピック分布を比較・検証してみたところ、プレー内容から推察される攻撃パターンとトピックの解釈に基づく攻撃パターンにある程度一致性があることが確認された。
さらに、文書条件付きトピック分布の類似度に基づいて類似プレーを抽出した結果、共通した攻撃パターンを持つ攻撃が複数抽出されたことが確認できた。

今後の課題として、はじめに、本研究で設けた仮定の妥当性について、さらなる検証を行いたい。
単語の羅列が実際に攻撃プレーを一意に定めうるのか、トピック混合状態が類似するプレーが実用上知りたい類似プレーになっているかなど、多くの適用により知見を深めたい。
そのうえで、モデルの数理的拡張として、以下の観点を今後の展望としたい。
まず、トピック数自体の推定も含めたモデルの検討が挙げられる。
現在、トピック数$K$を5つに固定して分析を回しているが、チーム戦術や対戦カードによりトピック数自体も異なると想定される。
そのため、AICやBICを指標としたモデル選択を行う、もしくはディリクレ過程を利用したトピック数の自動推定への応用が望まれる。
また、「試合終盤ではゲームが動きやすい」などのサッカーの特徴を反映するため、Dynamic topic model\cite{dtm}やTopics over time\cite{tot}など、トピックの時間変化を含めたモデルに拡張することも課題としてあげられる。
さらに、オフェンス時の選手配置がファーストディフェンスのやり方を規定すると考えられるため、攻撃に入る直前の守備配置を文書中の単語として採用するなど、文書作成における工夫も必要である。
最終的には、変数のスケールや離散化方法など調整を加えたうえで、今後蓄えられてくるであろうデータベースの有効利用のための推薦システムの構築を目指したい。


%謝辞
\subsection*{謝辞}
本研究で使用したデータはデータスタジアム株式会社様に提供いただきました。
また、本研究の貸与データは情報・システム研究機構の新領域融合研究プロジェクト『社会コミュニケーション』データ中心科学リサーチコモンズ事業『人間・社会データ』の支援を受けたものです。
ここに、感謝の意を表します。

%参考文献
\begin{thebibliography}{9}
\bibitem{foot}Football LAB(フットボールラボ)とは、Football LAB ~サッカーをデータで楽しむ~、http://www.football-lab.jp/pages/about/、2017年1月31日閲覧
\bibitem{lda} Blei, David M., Andrew Y. Ng, and Michael I. Jordan. "Latent dirichlet allocation." Journal of machine Learning research 3.Jan (2003): 993-1022.
\bibitem{topic_blue} 岩田具治:トピックモデル、講談社、2015.
\bibitem{topic_white} 奥村学:トピックモデルによる統計的潜在未解釈、コロナ者、2015.
\bibitem{2016nar} 成塚拓真、卯田純平、山崎義弘:サッカーの対戦的特徴に現れる普遍的な統計性の探求、統計数理研究所共同リポート363、スポーツデータ解析における理論と事例に関する研究集会、第3巻、pp. 83-90、2016.
\bibitem{dtm} Blei, David M., and John D. Lafferty. "Dynamic topic models." Proceedings of the 23rd international conference on Machine learning. ACM, 2006.
\bibitem{tot} Wang, Xuerui, and Andrew McCallum. "Topics over time: a non-Markov continuous-time model of topical trends." Proceedings of the 12th ACM SIGKDD international conference on Knowledge discovery and data mining. ACM, 2006.

\end{thebibliography}


\newpage
\end{multicols}
\end{document}
