\section{トピックモデルを用いた類似プレーの抽出手法}
\label{sec:lda}
本章ではトピックモデルを用いた類似プレーの抽出手法について説明する。
まず、提案手法を構築するにあたり必要となるトピックモデル、LDA(Latent Dirichlet Allocation)について概説した後、トピックモデルを用いてサッカーのプレー内容の分類およびそれに基づく類似プレーの抽出手法について説明する。



\subsection{Latent Dirichlet Allocation}
LDAは、もともと文書の確率的生成モデルとして提案されたモデルである。
ただし、文章の順序は無視し、Bag of Words(BoW)表現と呼ばれる単語と出現頻度のペアの集合をモデル化する。
ここで、BoW表現は単語が共起している現象を表しており、文書集合として考えると統計的に共起しやすい単語の集合がいくつか存在するはずである。
LDAは、この単語の共起性を統計モデルとして数理的に扱うために提案された。


\subsubsection{LDAの生成過程}
LDAでは、文書中の単語がどの洗剤トピックによって生成されたかを示す潜在編巣を導入する。
具体的には、文書$d$の$i$番目の単語を$w_{d,i}$として、対応する潜在変数を$z_{d,i}$と定義する。
洗剤トピックの添字集合を$\{1,2,...,K\}$とする($z_{d,i}\in\{1,2,...,K\}$)。
各潜在変数の値は、それぞれ単語の出現分布$\phi_k(k=1,2,...,K)$に対応する。
つまり、文書中の各単語は離散値をとる潜在変数を背後に保持しており、その潜在変数の値が同じ単語はトピックに属する(同じ単語の出現分布に従う)というモデル化を行う。

文書数を$M$、文書$d$の文章長(総単語数)を$n_d$とする。
LDAでは、文章は複数のトピックから構成され、その構成比を離散分布としてもつ。
$\theta_{d,k}$を、文章$d$でトピック$k$が出現する確率(文章$d$でのトピック$k$の構成比率)とし、トピック分布を$\bf{\theta}_d=(\theta_{d,1},...,\theta_{d,K})$とする。
$\theta_{d,v)$をトピック$k$における単語$v$の出現確率とし、単語の出現分布を$\bf{\theta}_k=(\theta_{k,1},\ldots,\theta_{k,V})$とする。
$\bf{\theta}_d$や$\phi_k$は確率ベクトルであるので、確率ベクトル上の確率分布であるDirichlet分布(Dirと表記)による生成を仮定する。
すなわち、
\begin{equation}
	\bvec{\theta}_d \sim Dir(\bvec{\alpha}) (d=1,\ldots,M)
    \phi_k \sim Dir(\bvec{\beta}) (k=1,\ldots,K)
\end{equation}

ここで、$\bvec{\alpha}=(\alpha_1,...,\alpha_K)$は$K$次元ベクトル、$\bvec{\beta}=(\beta_1,...,\beta_V)$は$V$次元ベクトルで、いずれもDirichlet分布のパラメータである。

単語$w_{d,i}$や潜在トピック$z_{d,i}$は離散値なので、多項分布(Multiと表記)を生成分布として仮定する。
すなわち、各文書$d(=1,...,M)$において、各単語は以下の生成過程を仮定する。
\begin{equation}
	z_{d,i} \sim Multi(\bvec{\theta}_d)
    w_{d,i} \sim Multi(\phi_{z_{d,i}}) (i=1,...,n_d)
\end{equation}

この生成過程のグラフィカルモデルを図\ref{fig:lda}に示す。



\subsubsection{LDAの学習アルゴリズム}
@@@@@@@@@@@@@@@@@@@@@@@@@@@@@@@@@@@@@@
@@@@@@@@@@@@@@@@@@@@@@@@@@@@@@@@@@@@@@
@@@@@@@@@@@@@@@@@@@@@@@@@@@@@@@@@@@@@@
@@@@@@@@@@@@@@@@@@@@@@@@@@@@@@@@@@@@@@
@@@@@@@@@@@@@@@@@@@@@@@@@@@@@@@@@@@@@@
@@@@@@@@@@@@@@@@@@@@@@@@@@@@@@@@@@@@@@
@@@@@@@@@@@@@@@@@@@@@@@@@@@@@@@@@@@@@@
@@@@@@@@@@@@@@@@@@@


\subsection{トピックモデルとしてみるサッカーのプレー}
自然言語処理における論理では、トピックモデルを適用する際に以下のような仮定をおいている。
\begin{itemize}
	\item 文はいくつかのトピック(話題)の重み付き和で構成されている
	\item トピックごとに使われやすい単語の分布がある
	\item 文は、トピック→単語の順番にサンプリングしたものである
	\item 文の中での単語の順番は不問(単語があれば意味のある文の復元は可能)
\end{itemize}

一方、本研究ではサッカーのプレーに対してトピックモデルを適用するにあたり、上記に対応するように以下の仮定を置いている。
\begin{itemize}
	\item ボール奪取時の状況やチームの特性に応じて、「一連の攻撃」はそれらから判断される、攻撃パターンの重み付き和で表現される
	\item 攻撃パターンごとに選択されやすいアクションや起こりやすい選手配置がある
	\item 一連のプレーは攻撃パターン→アクション/配置の順番にサンプリングしたものである
	\item 一連のプレー内部の順番を問わないことの意味
    \item 時系列データが格段に扱いやすくなる
    \item 単語を適切に設定すれば、実質的にプレー展開は定まる
\end{itemize}

ただし、そのための単語生成方法の検討が今後必要であることを先に述べる。


\subsection{類似手法の抽出手法}
トピックモデルを用いて攻撃パターンを分類し、その分類結果に基づく類似プレーの抽出手法を以下の通り提案する。
\begin{enumerate}
	\item 文書ごとに付加されるトピック分布より類似プレーを抽出する
	\item ある「一連の攻撃」$d_{req}$に関して、
	\item ボールタッチ/トラッキングデータより文書を作成する
	\item トピック分布$z_{req}=p(z|d_{req})$ を推定
	\item データベース中の攻撃データとコサイン類似度 $cos⁡(Z_{req},z_{db})$ を計算
	\item その他検索条件(攻撃長、特定プレーの存在など)に基づき、類似度の高いものから抽出
\end{enumerate}




