\section{はじめに}
\label{sec:hajimeni}

%%% 以下、まだ適当に。去年のレポートやパワポのコピペ。
数多くのスポーツで、センサやデータ計測員を駆使して多量のデータを取得し、統計的手法によってそれらを分析し、得られた示唆や知見をチーム戦略や選手評価に活用することが一般的となっている。
サッカーにおいても、ボール支配率やパス・シュート本数などの従来のスタッツを始め、ここ数年では選手の走った経路を逐次記録したトラッキングデータなどが計測されている\cite{foot}。


豊富なデータの中から、似たような「一連の攻撃」を探したいというタスクは、スタッフ、選手、観客などがプレーを振り返る際に重要である。
現在はデータ計測スタッフやマネージャー部員が逐一、「シュート」や「FK」というアクションタグや、「カウンター」や「クロス」などのプレー内容のフラグを付加している。
類似プレーを検索する際には、これらのタグ情報をもとに目的のプレーを抽出できることもあるが、それだけでは限界がある。
アクション以外にも「ピッチのどこで」「どんな選手配置の中」など一連の攻撃を説明するような検索条件はたくさん考えられるからである。
ただしそれらをひとつひとつ条件立てて検索するのは困難であり、類似プレー抽出の自動化が望まれる。


一連の攻撃を説明するような変数として、「起こったアクション」 「ボール位置」、「選手配置」など多様な変数が考えられる。
これらを変数としたクラスタリングを実施すれば類似プレーの抽出は可能であるが、同じ位置で、同じ順番に、同じ所要時間で展開するプレーはほぼ一つだけであり、複雑な時系列データすべてを見ても類似度は定義しにくい。
すなわち、スパース性の問題が生じる可能性がある。
また、時空間をある程度離散化してスパース性を解消しても、ひとつの攻撃の長さがそれぞれ異なるという問題が残り、異なる長さのベクトル間の類似度を測定することは難しい。

本研究では、トピックモデルを使用することで、長さの異なる文書でも、一つの共通したベクトルであるトピック分布として再表現できることに着目する。
すなわち、そのトピック分布を類似度に使うことで上記の問題を解決することができると考えられる。
本研究の目的は、トピックモデルを用いてサッカーの攻撃パターンの分類を行い、類似プレーの自動抽出を行うことである。
第\ref{sec:lda}章でトピックモデルの中でもLDA(Latent Dirichlet Allocation)について概説した後、第\ref{sec:input}章で分析に用いる変数を選定する。
第\ref{sec:exp}章で実際のプレーデータに対してLDAを適用し、推定されたトピックに関する解釈を行う。
最後に、本研究のまとめと今後の課題を第\ref{sec:owarini}章でまとめる。

