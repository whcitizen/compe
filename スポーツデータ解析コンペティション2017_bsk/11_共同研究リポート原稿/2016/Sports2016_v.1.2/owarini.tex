\section{おわりに}
\label{sec:owarini}

本研究では、トピックモデルを用いてサッカーの攻撃パターンを分類し、類似した一連の攻撃を自動で抽出する方法の提案を行った。
トピックモデルであるLDAを適用するにあたり、サッカーのトラッキングデータおよびボールタッチデータから一連の攻撃を説明するような単語の検討を行い、アクション名、ボール位置、選手配置(コンパクトネス × オフサイドライン)、守備脆弱度、ボール周辺選手配置の5種類の単語を作成し、攻撃内容を離散的な文書として表現した。
作成した文書に対してLDAを適用し、トピックごとの単語分布とデフォルト単語分布の比較、および単語条件付きトピック分布の参照を通じて各トピックの意味合いを解釈した結果、おおむねサッカーらしいトピック5つに自動分類できたと考えられる。
また、実際のプレー内容とトピック分布を比較・検証してみたところ、プレー内容から推察される攻撃パターンとトピックの解釈に基づく攻撃パターンにある程度一致性があることが確認された。
さらに、文書条件付きトピック分布の類似度に基づいて類似プレーを抽出した結果、共通した攻撃パターンを持つ攻撃が複数抽出されたことが確認できた。

今後の課題として、はじめに、本研究で設けた仮定の妥当性について、さらなる検証を行いたい。
単語の羅列が実際に攻撃プレーを一意に定めうるのか、トピック混合状態が類似するプレーが実用上知りたい類似プレーになっているかなど、多くの適用により知見を深めたい。
そのうえで、モデルの数理的拡張として、以下の観点を今後の展望としたい。
まず、トピック数自体の推定も含めたモデルの検討が挙げられる。
現在、トピック数$K$を5つに固定して分析を回しているが、チーム戦術や対戦カードによりトピック数自体も異なると想定される。
そのため、AICやBICを指標としたモデル選択を行う、もしくはディリクレ過程を利用したトピック数の自動推定への応用が望まれる。
また、「試合終盤ではゲームが動きやすい」などのサッカーの特徴を反映するため、Dynamic topic model\cite{dtm}やTopics over time\cite{tot}など、トピックの時間変化を含めたモデルに拡張することも課題としてあげられる。
さらに、オフェンス時の選手配置がファーストディフェンスのやり方を規定すると考えられるため、攻撃に入る直前の守備配置を文書中の単語として採用するなど、文書作成における工夫も必要である。
最終的には、変数のスケールや離散化方法など調整を加えたうえで、今後蓄えられてくるであろうデータベースの有効利用のための推薦システムの構築を目指したい。

