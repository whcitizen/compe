%%%メモ
%どういうトピックを抽出したいか←→トピックが似ているプレーの組み合わせを見てから類似プレーを定義

%パワポp11
\subsection{単語の作成}
\textcolor{red}{前述のように、}はじめにボールタッチデータから本研究で用いる単語を作成する必要がある。
ここでの目標は、サッカーの一連の攻撃で発生している複雑な動きを、なるべく端的な単語の羅列で表現することである。そして、作成された単語を並べた文をみれば、人やボールの時系列の移動やその順序が表現される状態を目指す。
サッカーの一連の攻撃において実際に生起している事象をもとに検討し(図 \figno{@@@@})、以下の5種類を作成することとした。
%図はパワポp11をあとでなんとかします

%パワポp20とp21
\subsection{推定されたトピックの解釈}
以上の設定のもとで推定された5つのトピックにおいて特徴的な単語を示す(表 \tabno{@@@@})。
この表は、トピックごとの単語分布$p(w|z_i),i={1,2,3,4,5}$と、デフォルト単語分布$p(w)$(すなわち、データ全体における正規化単語ヒストグラム:トピックが1つの場合の単語分布)との比を単語ごとに算出し、降順に並べたものである。
これに加え、単語条件付きトピック分布$p(z_i|w),i={1,2,3,4,5}$を参照しながら、各トピックの意味合いを解釈した(表\figno{@@@@})
%表は添付するパワポのp2、p3(p3はもとのp21と同一)

このように、作成した単語を用いてサッカーの攻撃プレーを表現した結果を、おおむねサッカーらしいトピック5つに自動分類できたと考えられる。
たとえば、トピック1はプレー位置が敵陣深くで、守備側のラインも深い。
さらに、ボール周囲に相手選手が多く、シュートやクロスも特徴的な単語となった。
また、トピック5ではボール位置が自陣深く、相手の守備ラインが高く脆弱度も高い一方で、スルーパスやシュート、被ファウルのような単語も特徴となった。
これらの結果は、攻撃をシュートで終えるためには、敵陣深くからのスルーパスやクロスまたはカウンターが有効であることを示唆している。
これは、一般に想定されるサッカーの知識と類似している。
また、サッカーの多くの時間はトピック4に表れる自陣でのビルドアップやその阻止、またトピック3に表れるアタッキングサード進入の攻防であることや、それ以外の時間が展開のない自陣でのボールポゼッションであることもうかがえる。

\subsection{実際の例}
%%%%%下に続くであろう類似プレー推薦に使う例を出したほうがいいと思うので、お任せします。
%イメージはもともとのパワポp22の下半分。

\subsection{今後の課題}

%課題とか
単語の生成方法:「そして、作成された単語を並べた文をみれば、人やボールの時系列の移動が結果的に表現されている状態を目指す」が実現されているか?

本研究:トピック5つの混合状態が似ていると類似→これはほんとうなのか?何が似ているのか?
