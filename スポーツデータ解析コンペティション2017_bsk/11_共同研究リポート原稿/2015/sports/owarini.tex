\section{おわりに}
\label{sec:owarini}

本研究では、統計的変化点検出手法であるChangeFinderをトラッキングデータに適用することで、サッカーの試合における戦況変化の抽出を行った。
ChangeFinderの入力変数の検討にあたっては、変数間の相関分析を通して共線性のない必要最低限な変数組となるよう配慮し、ボール位置、前線位置、コンパクトネス、守備脆弱度、攻撃率の5種類の指標を選定した。
松本山雅FC対湘南ベルマーレ戦後半のトラッキングデータにChangeFinderを適用した結果では、検出された合計約10か所の変化点に対応するように、その直前でパラメータ$\mu$の推定値$\hat{\mu}$の時系列的振る舞いに変動が確認された。
また、変化点スコア上昇の原因となるパラメータ推定値$\hat{\mu}$の変動から想定される戦況変化の内容と、実際のプレー内容がおおむね合致していることが確認できた。
本研究は、戦術変更が実戦況に効果を与えたか測定・分析するためのツールとして有用であると考えられる。

今度の課題として、実際の試合動画を専門的知見をもとに観察し、検出された戦況変化が妥当かつ十分なものであるか検証を進めたい。
また、より高度な戦術分析の実現に向け、VARモデルの$\mu$以外のパラメータについても精査し、変化点が検出された原因について分析を重ねる必要がある。
検出された変化点の前後でどのような戦況であったのか記述を行うことも今後の課題として挙げられる。
そして、データの背後に潜む戦況に関する潜在構造を推定し、戦況を変化させる因子の検出が可能となれば、より高度な戦術分析の実現へ向け前進するであろう。



