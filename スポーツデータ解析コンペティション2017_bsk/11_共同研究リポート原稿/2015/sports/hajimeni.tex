\section{はじめに}
\label{sec:hajimeni}
数多くのスポーツで、センサやデータ計測員を駆使して多量のデータを取得し、統計的手法によってそれらを分析し、得られた示唆や知見をチーム戦略や選手評価に活用することが一般的となっている。
サッカーにおいても、ボール支配率やパス・シュート本数などの従来のスタッツを始め、ここ数年では選手の走った経路を逐次記録したトラッキングデータなどが計測されている。
サッカーでは両チーム合わせ22人の選手とボールが互いに影響し合いながら45分ハーフを連続的にプレーしており、取得されるデータにはそのような複雑な動きがすべて記録されている。
サッカーの試合の大局を俯瞰的に把握するためには、この複雑なデータから選手や観戦者にとって解釈可能な形で示唆や知見を発掘する研究や分析が望まれる。

関連する研究の一つとして、Kijimaら\cite{kij}はサッカーの試合展開を支配する単純な法則性に言及している。
一見複雑に見える選手とボールの位置変化はフラクタル構造を持つというもので、サッカーの試合はある普遍的な法則に従うことを示唆している。

しかし一方で、観戦者や選手が感じる「戦況」は試合中に大きく変化することがある。
例えば、攻勢であったのにいつの間にか守勢に転じている、停滞していた状況が一気に動き出す、などである。
このような戦況変化は上記の不変性の崩れとも考えられ、戦況変化を適切に把握することができれば、試合を有利に進めることに繋がる。
しかし現状、実際に試合を観戦もしくはプレーすること以外に戦況変化を把握する方法が存在しない。
戦況変化を自動的に抽出できれば試合を有利に進めるための戦略が立てられるだけでなく、観戦者に対する情報提供などの面で有用であり、その研究意義は大きい。

戦況変化の自動抽出に向け、まずは戦況変化とは何かを検討する必要があるが、上述した例はいずれも選手やボールの時系列的な振る舞いが変化した結果であると解釈できる。
そのため、選手やボールに関するデータの時系列的振る舞いをモニタリングし、その振る舞いが変化した点を検出することで戦況の変化を抽出することが可能となると考えられる。
以上を踏まえると、統計的変化点検出\cite{yam_dm}の枠組みを使用し、データの時系列的な振る舞いの変化点を検出することでサッカーの試合における戦況変化の抽出を行うことができると考えられる。

なお、サッカーを始めとする集団競技の戦況は、選手個々の意思決定に依る動きだけでなく、チーム全体としての動き、もしくは選手相互間やボールとの関係から発生するものである。
したがって戦況が代表されるようなある一つの変数を見つけることは難しく、複数の変数および変数間の関係に着目して分析する必要がある。
また、実用を考える上で、リアルタイムに戦況変化が抽出されることが望ましい。
すると、多次元時系列データを対象に、確率モデルとして時系列モデルを仮定して、リアルタイムに変化点度合いのスコアを計算していく手法であるChangeFinder\cite{yam_cf}が適用可能であると考えられる。

本研究の目的は、トラッキングデータを用いてサッカーの試合における戦況変化の抽出を行うことである。
第\ref{sec:cf}章でChangeFinderについて説明した後、第\ref{sec:input}章で分析に用いる変数を選定する。第\ref{sec:exp}章で実データにChangeFinderを適用し、検出された戦況変化に関する解釈を行う。最後に、本研究のまとめと今後の課題を第\ref{sec:owarini}章でまとめる。

