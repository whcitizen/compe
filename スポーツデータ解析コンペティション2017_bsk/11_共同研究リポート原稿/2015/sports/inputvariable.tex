\section{入力変数の検討}
\label{sec:input}

\subsection{使用したデータの説明}
本研究で使用したデータは、2015明治安田生命J1リーグ1stステージ第2節鹿島アントラーズ対湘南ベルマーレ戦および第17節松本山雅FC対湘南ベルマーレ戦の計2試合に関して、1/30秒毎にパスやタックルなどボール周辺で発生したイベントおよびその発生時刻と位置を取得したボールタッチデータと、1/25秒毎に選手及び審判のピッチ上での位置を取得したトラッキングデータの2種類である。
なお、これらのデータはデータスタジアム株式会社から提供を受けたものである。


\begin{figure*}[htbp]
  \begin{center} %センタリングする
    \includegraphics[width=18cm]{img/variable.eps}
    \renewcommand{\baselinestretch}{1}%
    \caption{入力変数の概念図. (a)前線位置。両チームの選手がなす支配領域が均衡する前線の$X$座標に関する平均値。図中の白色破線に対応。(b)コンパクトネス。一番前方の選手と後方2番目の選手の$X$座標上の距離。図中の灰色と白色の四角形の幅にそれぞれ対応。(c)守備脆弱度。自軍のオフサイドラインより前方10m、後方5mの長方形のうち、最寄りの味方選手から5m以上離れており、最近傍選手が相手選手であるような地点の合計面積の割合。図中の灰色で囲まれている図形の合計面積が占める割合に相当。}
    \label{fig:var} %ラベルをつけ図の参照を可能にする
  \end{center}
\end{figure*}


\subsection{入力変数の検討}
VARモデルの入力変数として用いる指標について検討を行う。
まず、戦況を表すのに十分な入力変数を用意できるよう、試合全体の流れを表現すると思われる指標や、各チームの攻勢・守勢を表す指標を複数作成した。
その後、変数間の相関分析を行い、共線性のない必要最低限な変数組が選定できるように配慮した。
その結果、ボール位置、前線位置、コンパクトネス、守備脆弱度、攻撃率の5種類の指標を分析に用いる変数として選定した。

以下、それらの指標について詳細な説明を行う。特に、前線位置、コンパクトネス、守備脆弱度については図\ref{fig:var}に概念図を示している。
なお、提供を受けたデータは1/25秒や1/30秒間隔であったが、VARモデルへの適用を踏まえ、分析に使うデータはすべて1秒間隔となるよう加工した。
また、ピッチ中央を原点とし、コートの長辺方向を$X$方向、短辺方向を$Y$方向と定義する。




\subsubsection{ボール位置}
試合全体の流れを表現する指標としてボール位置を選定した。攻勢・守勢の切り替わりによるボールの上下動のみを考慮するとし、$X$座標のみを採用した。

\subsubsection{前線位置}
両チームの選手がなす支配領域が均衡する$X$座標を前線位置と定義した。
この変数は全選手の動きを代表するため,各チームの攻勢と守勢に合わせ値が増減すると予測される。
既往研究\cite{kij}に従い、以下のとおり作成した。
\begin{enumerate}
\renewcommand{\labelenumi}{\roman{enumi})}
\setlength{\itemsep}{-5pt}
 \item チームで正負の異なるガウスカーネル(標準偏差3m)を各選手が時刻$t$に位置する座標に設定する。
 \item 全てのカーネルを足し合わせ,ピッチ上で値が0となる線分を前線とする。
 \item 前線の$X$座標に対する平均値を算出し,これを時刻$t$における前線位置とする。
 \item i)~iii)を全時刻について算出する。
\end{enumerate}

\subsubsection{コンパクトネス}
ホーム・アウェイチーム毎に、「一番前方に存在する選手の$X$座標」と「後方より2番目に存在する選手の$X$座標」の距離をコンパクトネスと定義する。
各チームの選手がその時刻において、どれだけピッチ上で展開できているかを示す指標であり、攻撃を展開するとコンパクトネスは大きくなり、守備に回るとコンパクトネスは小さくなると予想される。

\subsubsection{守備脆弱度}
チームの守備力の脆弱性を表す指標として、自陣の最終ライン付近において相手選手が侵入している程度を守備脆弱度と定義する。
具体的には図\ref{fig:var}(c)に示すように、「自軍のオフサイドラインより前方10m、後方5mの長方形のうち、最寄りの味方選手から5m以上離れており、最近傍選手が相手選手であるような地点の合計面積の割合」である。
守勢への転換直後には守備脆弱度は上昇し、守備陣営が整うと守備脆弱度は減少すると予想される。
一方、カウンターや人数をかけた攻撃時には守備脆弱度は上昇することが予想される。

\subsubsection{攻撃率}
ホームチームの直近10分間における攻撃頻度を表す指標として攻撃率を定義する。
ボールタッチデータ中に離散的に取得されている各チームの攻撃アクションを内挿し、1秒毎の連続的な値として算出した。






