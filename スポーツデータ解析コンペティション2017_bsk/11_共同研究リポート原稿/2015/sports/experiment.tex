\section{戦況変化の抽出実験}
\label{sec:exp}

\subsection{適用条件}
ChangeFinderの入力変数として、ボール位置、前線位置、コンパクトネス(ホーム・アウェイチームごと)、守備脆弱度(ホーム・アウェイチームごと)、攻撃率の5種類・計7つのデータを7次元変数として設定した。
また、VARモデルの次数$K$は、オフラインでのVAR推定を行った場合の最小AIC値を参考に、$K=5$と決定した。
忘却率$r$や平滑化パラメータ$T,T′$についてはそれぞれ$r=0.01$、$T=50$、$T'=5$と設定した。
鹿島・湘南戦および松本・湘南戦の前後半のハーフ毎にChangeFinderを適用し、戦況変化の抽出を試みる。

なお、SDARアルゴリズムで用いるVARモデルの各種パラメータおよび統計量の初期値$\hat{\mu},\hat{\Sigma},C_i(i=1,...,K)$は、ハーフ開始60秒間で得られたデータから計算することとし、この間の変化点スコアは算出しない。

\subsection{適用結果および結果の考察}
松本・湘南戦後半を例として、図\ref{fig:res}の上段に入力時系列データ$\bvec{x}_t$を、下段にChangeFinderによる変化点スコアの出力値$Score(t)$を示す。
図\ref{fig:res}の上段において、1行目にボール位置と前線位置、2行目に湘南ベルマーレと松本山雅FCのコンパクトネス、3行目に湘南ベルマーレと松本山雅FCの守備脆弱度がそれぞれ灰色および黒色の実線で、そして4行目に松本山雅FCの攻撃率が示されている。
ボール位置と前線位置に関しては、正方向に値が上昇するにつれ、湘南ベルマーレの攻撃方向に進出していることを意味する。
下段は各時刻について算出された変化点スコア$Score(t)$であり、値が大きいほど戦況変化の度合いが大きいことを表している。

出力された変化点スコアを見ると、図中に(i)~(x)で示す合計10箇所で比較的大きな戦況変化が発生したと推定されたことが分かる。
紙面には掲載していないが鹿島・湘南戦前後半および松本・湘南戦前半の結果と比較すると、松本・湘南戦後半ではより高頻度で変化点が検出されていた。
松本・湘南戦後半では両チーム合わせ5つのゴールが生まれており、戦況の移り変わりが比較的大きかった試合であったことが、本結果において比較的高頻度に変化点が検出された理由として考えられる。

次に、検出された変化点が具体的にどういった戦況の変化を表しているのか、入力データの振る舞いと実際のプレーを比較することで結果の解釈および検証を行う。
ただし、図\ref{fig:res}からも分かるように入力データは複雑な動きをしているため、入力データから変化点検出の原因を直接分析するのは困難である。
そこで、パラメータ$\mu$の各時刻における推定値$\hat{\mu}$に着目する。
このパラメータはVARモデルの式(\ref{eq:var})より、自己回帰分を除いた平均的な値と解釈できる。
この推定値$\hat{\mu}$の振る舞いに着目し、検出された変化点との関係を分析することによって、どういった戦況の変化を表しているのか解釈を進めることが可能になると考える。

そこで、図\ref{fig:mu}に松本・湘南戦後半における推定値$\hat{\mu}$の推移と変化点の検出結果を示す。図中のグラフと変数の対応は図\ref{fig:res}と同様である。
図\ref{fig:res}で複雑な振る舞いをしていた入力値が、図\ref{fig:mu}の推定値$\hat{\mu}$では比較的滑らかな振る舞いをしていることが分かる。
また、推定値の時系列的振る舞いに変動があると、それ対応するように直後1分後あたりで変化点スコアが上昇していることが分かる。
なお、ChangeFinderでは第1段階および第2段階で移動平均処理を行っているため、観測データの入力から変化点の検出までに遅延が生じることに注意が必要である。

% 入力データと変化点スコアの出力
\begin{figure*}[htbp]
  \begin{center}
    \includegraphics[width=18cm]{img/res.eps}
    \renewcommand{\baselinestretch}{1}%
    \caption{松本・湘南戦後半における入力データおよび変化点スコア出力結果. 上段1行目にボール位置(ball)と前線位置(frontLine)、2行目に湘南ベルマーレと松本山雅FCのコンパクトネス(compact)、3行目に湘南ベルマーレと松本山雅FCの守備脆弱度(defence)をそれぞれ灰色および黒色で、4行目に松本山雅FCの攻撃率(attack)を示す。AWAYおよびHOMEはそれぞれ湘南ベルマーレと松本山雅FCに対応する。下段はChangeFinderによる変化点スコアの出力値$Score(t)$である。}
    \label{fig:res}
  \end{center}
\end{figure*}

%平均μと変化点スコアの出力および因果関係の考察
\begin{figure*}[htbp]
  \begin{center}
    \includegraphics[width=18cm]{img/mu.eps}
    \renewcommand{\baselinestretch}{1}%
    \caption{松本・湘南戦後半におけるパラメータ$\mu$の推定値$\hat{\mu}$の推移および変化点スコア出力結果. 図中のグラフと変数の対応は図\ref{fig:res}に同じ。なお、試合開始60秒間のパラメータは推定されていない。}
  \label{fig:mu}
  \end{center}
\end{figure*}

表\ref{tbl}に、各戦況変化点(i)~(x)について代表的な$\hat{\mu}$の変動の様子およびそれから想定される戦況変化の内容、ならびに実際にボールタッチデータから確認されたプレーの内容をまとめた。
例えば戦況変化点(i)の直前では、ボールが湘南攻撃方向に移動し、松本の守備脆弱度が上昇するという変動がパラメータ推定値$\hat{\mu}$の中で確認された。
この結果より、湘南が攻勢へと転換したことが戦況変化の内容として想定される。
実際にボールタッチデータより、湘南が複数のパスを繋ぎながら攻撃を展開している様子が確認できた。
他の変化点についても、表中にあるように、変化点スコア上昇の原因となるようなパラメータ推定値$\hat{\mu}$の変動から想定される戦況変化の内容が、実際のプレー内容と概ね合致していることが確認できた。

\begin{table*}[t]
  \centering
  \caption{検出された変化点の解釈および実際のプレーとの関係}
  \label{tbl}
  \begin{tabular}{|c|l|l|l|}
  \hline
  \# & 代表的な$\hat{\mu}$の変動 & 想定される戦況変化 & 実際に確認されたプレー  \\ \hline \hline
  i & \begin{tabular}[c]{@{}l@{}}ボールが湘南攻撃方向に移動\\ 松本の守備脆弱度が上昇\end{tabular} & 湘南の攻勢への転換 & 湘南がパスを繋ぎ攻撃を展開 \\ \hline
  i\hspace{-.1em}i & \begin{tabular}[c]{@{}l@{}}ボールが松本攻撃方向に移動\\ 湘南のコンパクトネスが低下\\ 松本のコンパクトネスが上昇\\ 湘南の守備脆弱度が上昇\\ 松本の守備脆弱度が低下\end{tabular} & 松本の攻勢への転換 & 松本がPKを獲得 \\ \hline
  i\hspace{-.1em}i\hspace{-.1em}i & \begin{tabular}[c]{@{}l@{}}湘南のコンパクトネスが上げ止まり\\ 松本のコンパクトネスが下げ止まり\\ 松本の守備脆弱度が回復\end{tabular} & 松本の守勢の実現 & 湘南がFKを獲得し松本が守備体勢に \\ \hline
  i\hspace{-.1em}v & \begin{tabular}[c]{@{}l@{}}ボールが松本攻撃方向に回復\\ 松本のコンパクトネスが上昇\end{tabular} & 松本の攻勢への転換 & 松本が自陣でFKを獲得しピンチを脱する \\ \hline
  v & \begin{tabular}[c]{@{}l@{}}松本のコンパクトネスが低下\\ 松本の守備脆弱度が上昇\\ 湘南の守備脆弱度が低下\end{tabular} & \begin{tabular}[c]{@{}l@{}}湘南の攻勢への転換\\ 松本の守勢の実現\end{tabular} & 湘南のCK/FKによる連続的な攻撃 \\ \hline
  v\hspace{-.1em}i & 湘南の守備脆弱度が回復 & 湘南の守備の実現 & 湘南得点後の松本キックオフ \\ \hline
  v\hspace{-.1em}i\hspace{-.1em}i  & 松本の守備脆弱度が低下 & 松本の守勢の実現 & 複数選手の交代により松本守勢の立て直しか \\ \hline
  v\hspace{-.1em}i\hspace{-.1em}i\hspace{-.1em}i & ボール位置が短時間に前後 & 試合の活性化 & 松本得点直後に湘南の得点 \\ \hline
  i\hspace{-.1em}x & \begin{tabular}[c]{@{}l@{}}ボールが松本攻撃方向に移動\\ 湘南の守備脆弱度が上昇\end{tabular} & 松本の攻勢への転換 & 松本のシュートを含む試合展開 \\ \hline
  x & \begin{tabular}[c]{@{}l@{}}ボールが松本攻撃方向に移動\\ 湘南の守備脆弱度が上昇\end{tabular} & 松本の攻勢への転換 & 同点弾を狙った松本の攻撃 \\ \hline
  \end{tabular}
\end{table*}

以上の結果を踏まえると、本分析方法により、攻勢・守勢への転換やセットプレーによる連続攻撃、連続得点による試合の活性化など、想定される戦況変化は概ね検出できていると考えられる。
ただし、実際の試合映像などを確認し、検出された戦況変化が妥当なものであるのか、また、未検出となった戦況変化等が存在しないかを専門的知見と共に検討を重ねたい。
また、現在は検出された変化点との関係をパラメータ$\mu$のみで見ているが、戦況のより高度な解釈に向け、$\mu$以外のパラメータについても精査を行う必要がある。
さらに、検出された戦況変化に対応するプレーにはセットプレーや選手交代などアウトオブプレーが多く含まれている。
実際、セットプレーを機に戦況が大きく変わることは多々あるが、変化点(i)や(x)などインプレー中の戦況変化の抽出も重要であろう。
インプレー中の戦況変化の抽出のためには、ChangeFinderに適用する時系列データからアウトオブプレー中のデータを排除する、一つのインプレーデータ毎にChangeFinderを適用するなど、分析方法を修正することが解決策として考えられる。



